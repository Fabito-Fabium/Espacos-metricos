

%% Classe de documento
\documentclass[%% Opções (^ = padrão; > = para pacotes):
  a4paper,%% Tamanho de papel: a4paper, letterpaper (^), etc.
  12pt,%% Tamanho de fonte: 10pt (^), 11pt, 12pt, etc.
  fleqn,%% Alinhamento de equações à esquerda (centralizado por padrão)
%   draft,%% Aparência de documento (>): draft ou final (^)
  english,%% Idioma secundário (penúltimo) (>)
  brazilian,%% Idioma primário (último) (>)
]{article}

%% Pacotes utilizados
\usepackage[%% Opções (^ = padrão; ¹ = Leiaute A; ² = Leiaute B):
%   Layout    = A,%% Leiaute (margens): A (esquerda menor + recuo) ou B (iguais) (^)
%   Font      = Calibri,%% Fonte principal: Arial (²), Calibri (¹), Times ou LaTeX
%   RuleWidth = Text,%% Largura de linha (cabeçalho e rodapé): Zero (²), Text ou Paper (¹)
%   AffilIcon = Off,%% Ícones de e-mail, Lattes, ORCID e Instituição: On (^) ou Off
%   DocLic    = None,%% Licença do documento: CC (Creative Commons) (^) ou None
%   CCLic     = BY-NC-ND,%% Licença CC: BY (^), BY-SA, BY-ND, BY-NC, BY-NC-SA ou BY-NC-ND
%   Link      = TextColor,%% Cor de hyperlinks: DarkBlue (^) ou TextColor
%   PageNum   = Off,%% Numeração de páginas: On (^) ou Off
%   SectNum   = Off,%% Numeração de seções (\section): On (²) ou Off (¹)
%   SectUnnum = Center,%% Alinhamento de seções (\section*): Center (¹) ou Left (²)
%   Caption   = Left,%% Alinhamento de legendas: Center (^) ou Left
%   Source    = Left,%% Alinhamento de fonte (referência): Center (^) ou Left
%   ABNTCit   = NSB,%% Citação ABNT: AAY (NOME, ANO) (^), NRB (1) ou NSB [1]
%   BibDOI    = Icon,%% Ícone de DOI em referências: Icon ou Name (^)
%   BibURL    = Icon,%% Ícone de URL em referências: Icon ou URL (^)
]{utfpr-article}

\usepackage[hyperlink]{qrcode}
\usepackage{environ}
\usepackage{tabularray}
\usepackage[fleqn]{nccmath}
\usepackage{bm}
\relpenalty=9999
\binoppenalty=9999
\NewEnviron{ceqnalign*}
    {\vspace{3mm}
    \begin{ceqn}
        \begin{align*}
            \BODY
        \end{align*}
\end{ceqn}}
\NewEnviron{ceqnalign}[1]
{\vspace{3mm}
	\begin{ceqn}
		\begin{align*}
			\BODY \tag{#1}
		\end{align*}
\end{ceqn}}
\usepackage{catchfilebetweentags}
\DeclareMathOperator{\Ima}{Im}
\newcommand{\Laplace}{\mathscr{L}}
\newcommand{\bN}{\mathbb N}
\newcommand{\inN}{\in \bN}
\newcommand{\dotline}{\\.\dotfill.\\}
\newcommand{\hlinear}{\\.\hspace{0pt}\hrulefill.\\}
\newcommand{\sqn}[1]{(#1_n)_{n \in \bN}}
\newcommand{\fmm}[3]{\{#1_#2\}_{#2 \in #3}}
\newcommand{\seq}[3]{(#1_#2)_{#2 \in #3}}
\newcommand{\dfunc}[3]{#1: #2 \rightarrow #3}
\newcommand{\tozero}[1]{#1 \rightarrow 0}
\newcommand{\toinf}[1]{#1 \rightarrow \infty}
\newcommand{\limsq}[3]{\lim_{#1 \rightarrow #2} #3_#1}
\newcommand{\limif}[2]{\lim_{#1 \rightarrow \infty} #2}
\newcommand{\limzr}[2]{\lim_{#1 \rightarrow 0} #2}
\newcommand{\bC}{\mathbb C}
\newcommand{\bZ}{\mathbb Z}
\newcommand{\bR}{\mathbb R}
\newcommand{\bRn}{\mathbb R^n}
\newcommand{\bQ}{\mathbb Q}
\newcommand{\itr}[1]{\text{int}(#1)}
\newcommand{\parth}[1]{\left(#1\right)}
\newcommand{\chavs}[1]{\left\{#1\right\}}
\newcommand{\mdl}[1]{\left|#1\right|}
\newcommand{\xssy}[2]{\left(#1 - #2, #1 + #2\right)}
\newcommand{\floor}[1]{\left\lfloor#1\right\rfloor}
\newcommand{\ceil}[1]{\left\lceil#1\right\rceil}
\newcommand{\asbeps}[2]{|#1 - #2| < \epsilon}
\newcommand{\iffc}[1]{\overset{#1}{\iff}}
\newcommand{\impliesc}[1]{\overset{#1}{\implies}}
\newcommand{\eqc}[1]{\overset{#1}{=}}
\newcommand{\gec}[1]{\overset{#1}{\ge}}
\newcommand{\lec}[1]{\overset{#1}{\le}}
\newcommand{\sen}{\mathrm{sen}}

%%%% Autor(es) (de 1 a 8, Leiaute A; de 1 a 16, Leiaute B): {Número}; {Dados}
 \Author{1}{%% Bolsista ou voluntário(a) principal (primeiro)
   Name        = {Fabio Zhao Yuan Wang},%
   Email       = {fabioyuan@gmail.com},%
 }
% \Author{2}{%% Colaborador(a) (segundo ao penúltimo)
%   Name        = {Nome Completo do{(a)} Autor{(a)}-A2},%
%   Email       = {author2@domain},%
%   Lattes      = {0000000000000002},%% Opcional
%   ORCID       = {0000-0000-0000-0002},%% Opcional (CHKTEX 8)
%   Affiliation = {Instituição (Nome por Extenso), Cidade, Estado, País},%
%   Role        = {Discente de Nome do Curso},%% Leiaute B
% }
% \Author{3}{%% Orientador(a) principal (último)
%   Name        = {Nome Completo do{(a)} Autor{(a)}-A3},%
%   Email       = {author3@domain},%
%   Lattes      = {0000000000000003},%% Opcional
%   ORCID       = {0000-0000-0000-0003},%% Opcional (CHKTEX 8)
%   Affiliation = {\UTFPRName, Cidade, Paraná, Brasil},%
%   Role        = {Docente do Nome do Departamento ou Programa},%% Leiaute B
% }
%%%% Digital Object Identifier (DOI): {Name}
% \DOIName{10.1000/xyz123}
%%%% Datas: [Ano] (atual por padrão; opcional); {Recebido}; {Aprovado}
% \Dates[2023]{DD mmm. \YearNum}{DD mmm. \YearNum}
%%%% Evento (SEI, SICITE, etc.): {Dados}
% \EventInfo{%
%   Name    = {Nome Completo do Evento},%
%   Acronym = {EVNT \YearNum},%
%   Date    = {DD a DD de Mmmmmm de \YearNum},%
%   Local   = {Cidade, Paraná, Brasil},%
%   Logo    = {example-image},%% Arquivo de imagem em ./Logos/
%   Header  = {example-image},%% Arquivo de imagem em ./Headers/
%   URL     = {https://www.example.com/},%
% }
%%%% Instituição (se nenhum evento): {Dados}
 \InstitutionInfo{%
   Campus     = {Curitiba},%
   Department = {Departamento de Eletrônica},%
 }


%% Arquivo(s) de referências
\addbibresource{utfpr-article.bib}

%% Início do documento
\begin{document}
Considere o espaço euclidiano $\bRn$. Os pontos de $\bRn$ são listas $x=(x_1, \cdots, x_n)$ onde $x_i \in\bR$
para todo $i \inN_{\le n}$. Há três maneiras de definir distância entre dois pontos em $\bRn$. Dados 
$x=(x_1, \cdots, x_n)$ e $y=(y_1, \cdots, y_n)$, 

\begin{ceqnalign*}
  &d(x,y) = \parth{\sum_{i=1}^{n} (x_i - y_i)^2}^{\frac{1}{2}},
  &d'(x,y) = \sum_{i=1}^{n} |x_i - y_i|,
  &&d''(x,y) = \max_{i\inN_{\le n}}|x_i - y_i|.
\end{ceqnalign*}
\ExecuteMetaData[Section/sol.tex]{prop1}
Daqui, vejamos outros exemplos de espaços métricos.\\
\textbf{Exemplo 1:} Seja $X$ um conjunto arbitrário. Uma função real $f: X \rightarrow \bR$ chama-se
\textit{limitada} quando existe uma constante $k > 0$ tal que $|f(x)| \le k$ para todo $x \in X$. 
Note que a soma e o produto de funções limitadas são também limitadas. Denotando 
o conjunto das funções limitadas que vão de $X$ a $\bR$ como $\mathcal B (X; \bR)
$, é possível definirmos uma métrica em $\mathcal B (X; \bR)$ tal que, para $f, g 
\in \mathcal B (X; \bR)$,
\begin{ceqnalign*}
  d(f, g) = \sup_{x \in X} |f(x) - g(x)|. \tag{m1}
\end{ceqnalign*}
Esta é a chamada \textit{métrica da convergência uniforme}, ou \textit{métrica do sup}.

Note que, para $X = \bN_{\le n}$ e $f, g: X \rightarrow \bR$ limitadas, é possível construir as sequências
$\seq x k {\bN_{\le n}}$ e $\seq y k {\bN_{\le n}}$, tais que para todo $k \in \bN_{\le n}$, 
temos que $x_k = f(k)$ e $y_k = g(k)$. Deste modo, se temos $x, y \in \bRn$ tais que a $i$-ésima coordenada
de $x$ equivale a $x_i$, e análogamente para $y$, tem-se que (m1) pode ser reduzida a métrica $d''$ de $\bRn$.\\
\textbf{Exemplo 2: } \textit{(Espaços vetoriais normados)} Considere um espaço vetorial $E$ sobre o corpo $\bR$.
Uma \textit{norma} em $E$ é uma função $|\quad|: E \rightarrow \bR$, que a cada vetor $x \in E$ é associado
o número real $|x|$, denominado \textit{norma de x}, de modo a serem cumpridas as seguintes condições 
para cada $u, v \in E$ e $\lambda \in \bR$:
\begin{enumerate}[label=N\arabic*)]
  \item Se $x \neq 0_{E}$, então $|x| \neq 0_{\bR}$;
  \item $|\lambda \cdot x| = |\lambda| |x|$;
  \item $|x+y| \le |x| + |y|$.
\end{enumerate}

Um \textit{espaço vetorial normado} é um par $(E, $\nrm$)$ onde $E$ é um espaço vetorial sobre o corpo 
dos reais e $\nrm$ é uma \textit{norma} em $E$. Exemplos de espaços vetoriais normados são 
$(\bRn, \nrm)$, $(\bRn, \nrm')$ e $(\bRn, \nrm'')$, onde, para 
$x=(x_1, \cdots, x_n) \in \bRn$, se tem

\begin{ceqnalign*}
  |x|=d(x, 0_{\bRn}), \quad |x|' = d'(x, 0_{\bRn}), \quad |x|'' = d''(x, 0_{\bRn})
\end{ceqnalign*}

Outro exemplo de espaço vetorial normado é $\mathcal B (X, \bR)$, onde $||f|| = \sup_{x \in X} |f(x)|$. Note
que foi utilizado a notação $||f||$ para denotar a norma da função $f$, a fim de não confundir com a função
$|f|: X \rightarrow \bR$, tal que $|f|(x) = |f(x)|$, a qual chamamos de '\textit{função módulo de $f$}'.

Todo espaço vetorial normado $(E, \nrm)$ torna-se um espaço métrico por meio da definição $d(x,y) = |x - y|$.
Esta métrica diz-se \textit{proveniente da norma} $\nrm$. Por exemplo, as métricas $d$, $d'$ e $d''$ em
$\bRn$ são provenientes das normas $\nrm$, $\nrm'$ e $\nrm''$, respectivamente. Também, a métrica do
sup em $\cB(X;\bR)$ é proveniente da norma que acabamos de introduzir neste espaço. Podemos então escrever 
$||f-g||$ em vez de $d(f,g)$.

Note que as propriedades de uma métrica que provém de uma norma resultam imediatamente das análogas para a 
norma. Por exemplo, a desigualdade triangular é obtida através de (N3).

\textbf{Exemplo 3:} \textit{(Espaços vetoriais com produto interno)} Seja $E$ um espaço vetorial real. Um 
\textit{proudto interno} em $E$ é uma função $\fintpr: E \times E \rightarrow \bR$, que associa a cada 
par ordenado de vetores $x, y \in E$ um número real $\intpr{x, y}$ chamado o produto interno de $x$ por $y$,
de modo a serem cumpridas as condições abaixo, para $x, x', y \in E$ e $\lambda \in \bR$, temos: 

\begin{enumerate}[label=P\arabic*)]
  \item $\intpr{x+x', y} = \intpr{x,y} + \intpr{x',y}; $
  \item $\intpr{\lambda x, y} = \lambda \intpr{x, y}; $
  \item $\intpr{x, y}=\intpr{y,x}; $
  \item $x\neq 0 \implies \intpr{x,x}>0. $
\end{enumerate}

A partir do produto interno, define-se a norma de um vetor $x\in E$ pondo $|x|=\sqrt{\intpr{x,x}}$, ou seja,
$|x|^2=\intpr{x,x}$. As propriedades $N1$ e $N2$ são imediatas. Quanto a $N3$, ela decorre da 
\textit{Desigualdade de Cauchy-Schwarz,} $|\intpr{x,y}| \le |x| \cdot |y|$. 

Para a demonstração da forma geral da Desigualdade de Cauchy-Schwarz, isto é, a Desigualdade de Hölder,
considere os seguintes lemas:
\hlinear
\textbf{Lema 1: (Desigualdade de Young)} Para qualquer $a, b$ não-negativos, e sejam $p$ e $q$ expoentes 
conjugados, isto é,

\begin{ceqnalign*}
  \frac{1}{p} + \frac{1}{q} = 1, \quad \text{onde, $p, q > 1$},
\end{ceqnalign*}
temos que 
\begin{ceqnalign*}
  ab \le \frac{a^p}{p} + \frac{b^q}{q}.
\end{ceqnalign*}
\dotline
\textbf{Dem:} Considere a função $f(x) = x^p$. Note que, como $p > 1$, então $f''(x) = p(p-1)x^{p-2}$. 
Se $x$ é não-negativo, então $x^{p-2}$ também é não-negativo. Mais ainda, 
\begin{ceqnalign*}
  p > 1 \implies (p - 1) > 0 \implies p(p-1)>0,
\end{ceqnalign*}
sendo assim, $f''(x) \ge 0$. Deste modo, temos que $f(x)$ é uma função convexa para todo $x$ não-negativo.
Das propriedades de funções convexas, temos que, para quaisquer $x, a \in \bR_{\ge 0}$,
\begin{ceqnalign*}
  f(x) \ge f(a) + f'(a)(x-a).
\end{ceqnalign*}
Com isto, seja $a = 1$,
\begin{ceqnalign*}
  f(x) \ge f(1) + f'(1)(x-1) = 1^p + p(1)^{p-1}(x-1) = 1 + p(x-1),
\end{ceqnalign*}
ou seja,
\begin{ceqnalign*}
  x^p \ge 1 + p(x-1), \quad x\in \bR_{\ge 0}, \quad p>1.
\end{ceqnalign*}
Visto que, por hipótese, $a$ e $b$ são não-negativos, então $x = ab^{1-q} \ge 0$, deste modo,
\begin{ceqnalign*}
  \parth{ab^{1-q}}^p &\ge 1 + p(ab^{1-q}-1), \\
  \frac{a^pb^{(1-q)p}}{p} &\ge \frac{1}{p} + (ab^{1-q}-1) \implies 
  \frac{a^pb^{-q}}{p} \ge \frac{1}{p} + (ab^{1-q} - 1),\\
  \frac{a^p}{p} &\ge b^{q}\parth{\frac{1}{p} - 1} + ab \implies 
  \frac{a^p}{p} + \frac{b^{q}}{q} \ge ab. 
\end{ceqnalign*}
Com isto, temos o que queríamos. $\square$
\hlinear
\textbf{Lema 2: } Sejam $p$ e $q$ expoentes conjugados e considere duas sequências $(x_i)_{i \inN_{\le n}}$ e $(y_i)_{i \inN_{\le n}}$,
tais que,

\begin{ceqnalign*}
  \sum_{i = 1}^n |x_i|^p = 1, \quad\quad  \sum_{i = 1}^n |y_i|^q = 1,
\end{ceqnalign*}
então, 

\begin{ceqnalign*}
  \sum_{i = 1}^n |x_i \cdot y_i| \le 1.
\end{ceqnalign*}
\dotline
\textbf{Dem:} Ora, da Desigualdade de Young, como $|x_i| \ge 0$ e $|y_i| \ge 0$ para todo $i \inN_{\le n}$,
então,

\begin{ceqnalign*}
  |x_i|\cdot |y_i| \le \frac{|x_i|^p}{p} + \frac{|y_i|^q}{q}, \quad \quad \forall i \inN_{\le n}
\end{ceqnalign*}
deste modo,
\begin{ceqnalign*}
  \sum_{i=1}^{n} |x_i y_i| \le \sum_{i=1}^{n} \parth{\frac{|x_i|^p}{p} + \frac{|y_i|^q}{q} }=
  \frac{1}{p}\sum_{i=1}^{n} |x_i|^p + \frac{1}{q}\sum_{i=1}^n |y_i|^q = \frac{1}{p} + \frac{1}{q} = 1,
\end{ceqnalign*}
como queríamos. $\square$
\hlinear
\textbf{Teorema: } (Desigualdade de Hölder) Sejam $p$ e $q$ expoentes conjugados e considere as sequências 
$x= \sqnn x i$ e $y = \sqnn y i$. Então, 
\begin{ceqnalign*}
  \sum_{i = 1}^{n} \mdl{x_i y_i} \le \parth{\sum_{i = 1}^{n} |x_i|^p}^{1/p}
  \parth{\sum_{i = 1}^{n} |y_i|^q}^{1/q}
\end{ceqnalign*}
\dotline
\textbf{Dem:} Se $x$ ou $y$ são nulos, temos o que queríamos. Considere $x$ e $y$ não-nulos e as 
sequências $\sqnn \alpha i$ e  $\sqnn \beta i$ tais que $\alpha_i = \mdl{\frac{x_i}
{\parth{\sum_{w=1}^\infty |x_w|^p}^{1/p}}}$
e $\beta_i = \mdl{\frac{y_i}{\parth{\sum_{w=1}^\infty |y_w|^q}^{1/q}}}$. Com isto, temos que 
\begin{ceqnalign*}
  \sum_{i=1}^n \alpha_i^p = 1, \quad\quad \sum_{i=1}^n \beta_i^q = 1.
\end{ceqnalign*}
Deste modo, do \textbf{Lema 2}, 
\begin{ceqnalign*}
  \sum_{i=1}^n \alpha_i\beta_i \le 1 
\end{ceqnalign*}
ou seja, 
\begin{ceqnalign*}
  \sum_{i = 1}^n \parth{\mdl{\frac{x_i} {\parth{\sum_{w=1}^n |x_w|^p}^{1/p}}} 
  \mdl{\frac{y_i}{\parth{\sum_{w=1}^\infty |y_w|^q}^{1/q}}}} \le 1 \\
  \sum_{i = 1}^{n} \mdl{x_i y_i} \le \parth{\sum_{i = 1}^{n} |x_i|^p}^{1/p}
  \parth{\sum_{i = 1}^{n} |y_i|^q}^{1/q}.
\end{ceqnalign*}
como queríamos. $\square$
\hlinear

Note que, para demonstrar a Desigualdade de Cauchy-Schwarz, basta utilizar a Desigualdade de Hölder com 
$p = q = 2$ e em seguida aplicar a desigualdade triangular.

Mais ainda, para o caso $\toinf n$, a Desigualdade de Hölder e o \textbf{Lema 2} são satisfeitos quando
as sequências $x$ e $y$ são elementos de $\ell^w$, onde $w = \max\{p, q\}$.
\hlinear
\textbf{Teorema: }(Desigualdade de Minkowski) Seja $p > 1$ e considere as sequências $x$ e $y$ em $l^p$,
então, 

\begin{ceqnalign*}
  \parth{\sum_{i=1}^\infty |x_i + y_i|^p}^{1/p} \le \parth{\sum_{i=1}^\infty |x_i|^p}^{1/p} +
  \parth{\sum_{i=1}^\infty |y_i|^p}^{1/p}
\end{ceqnalign*}
\end{document}
